% Options for packages loaded elsewhere
\PassOptionsToPackage{unicode}{hyperref}
\PassOptionsToPackage{hyphens}{url}
%
\documentclass[
]{article}
\usepackage{amsmath,amssymb}
\usepackage{iftex}
\ifPDFTeX
  \usepackage[T1]{fontenc}
  \usepackage[utf8]{inputenc}
  \usepackage{textcomp} % provide euro and other symbols
\else % if luatex or xetex
  \usepackage{unicode-math} % this also loads fontspec
  \defaultfontfeatures{Scale=MatchLowercase}
  \defaultfontfeatures[\rmfamily]{Ligatures=TeX,Scale=1}
\fi
\usepackage{lmodern}
\ifPDFTeX\else
  % xetex/luatex font selection
\fi
% Use upquote if available, for straight quotes in verbatim environments
\IfFileExists{upquote.sty}{\usepackage{upquote}}{}
\IfFileExists{microtype.sty}{% use microtype if available
  \usepackage[]{microtype}
  \UseMicrotypeSet[protrusion]{basicmath} % disable protrusion for tt fonts
}{}
\makeatletter
\@ifundefined{KOMAClassName}{% if non-KOMA class
  \IfFileExists{parskip.sty}{%
    \usepackage{parskip}
  }{% else
    \setlength{\parindent}{0pt}
    \setlength{\parskip}{6pt plus 2pt minus 1pt}}
}{% if KOMA class
  \KOMAoptions{parskip=half}}
\makeatother
\usepackage{xcolor}
\usepackage[margin=1in]{geometry}
\usepackage{color}
\usepackage{fancyvrb}
\newcommand{\VerbBar}{|}
\newcommand{\VERB}{\Verb[commandchars=\\\{\}]}
\DefineVerbatimEnvironment{Highlighting}{Verbatim}{commandchars=\\\{\}}
% Add ',fontsize=\small' for more characters per line
\usepackage{framed}
\definecolor{shadecolor}{RGB}{248,248,248}
\newenvironment{Shaded}{\begin{snugshade}}{\end{snugshade}}
\newcommand{\AlertTok}[1]{\textcolor[rgb]{0.94,0.16,0.16}{#1}}
\newcommand{\AnnotationTok}[1]{\textcolor[rgb]{0.56,0.35,0.01}{\textbf{\textit{#1}}}}
\newcommand{\AttributeTok}[1]{\textcolor[rgb]{0.13,0.29,0.53}{#1}}
\newcommand{\BaseNTok}[1]{\textcolor[rgb]{0.00,0.00,0.81}{#1}}
\newcommand{\BuiltInTok}[1]{#1}
\newcommand{\CharTok}[1]{\textcolor[rgb]{0.31,0.60,0.02}{#1}}
\newcommand{\CommentTok}[1]{\textcolor[rgb]{0.56,0.35,0.01}{\textit{#1}}}
\newcommand{\CommentVarTok}[1]{\textcolor[rgb]{0.56,0.35,0.01}{\textbf{\textit{#1}}}}
\newcommand{\ConstantTok}[1]{\textcolor[rgb]{0.56,0.35,0.01}{#1}}
\newcommand{\ControlFlowTok}[1]{\textcolor[rgb]{0.13,0.29,0.53}{\textbf{#1}}}
\newcommand{\DataTypeTok}[1]{\textcolor[rgb]{0.13,0.29,0.53}{#1}}
\newcommand{\DecValTok}[1]{\textcolor[rgb]{0.00,0.00,0.81}{#1}}
\newcommand{\DocumentationTok}[1]{\textcolor[rgb]{0.56,0.35,0.01}{\textbf{\textit{#1}}}}
\newcommand{\ErrorTok}[1]{\textcolor[rgb]{0.64,0.00,0.00}{\textbf{#1}}}
\newcommand{\ExtensionTok}[1]{#1}
\newcommand{\FloatTok}[1]{\textcolor[rgb]{0.00,0.00,0.81}{#1}}
\newcommand{\FunctionTok}[1]{\textcolor[rgb]{0.13,0.29,0.53}{\textbf{#1}}}
\newcommand{\ImportTok}[1]{#1}
\newcommand{\InformationTok}[1]{\textcolor[rgb]{0.56,0.35,0.01}{\textbf{\textit{#1}}}}
\newcommand{\KeywordTok}[1]{\textcolor[rgb]{0.13,0.29,0.53}{\textbf{#1}}}
\newcommand{\NormalTok}[1]{#1}
\newcommand{\OperatorTok}[1]{\textcolor[rgb]{0.81,0.36,0.00}{\textbf{#1}}}
\newcommand{\OtherTok}[1]{\textcolor[rgb]{0.56,0.35,0.01}{#1}}
\newcommand{\PreprocessorTok}[1]{\textcolor[rgb]{0.56,0.35,0.01}{\textit{#1}}}
\newcommand{\RegionMarkerTok}[1]{#1}
\newcommand{\SpecialCharTok}[1]{\textcolor[rgb]{0.81,0.36,0.00}{\textbf{#1}}}
\newcommand{\SpecialStringTok}[1]{\textcolor[rgb]{0.31,0.60,0.02}{#1}}
\newcommand{\StringTok}[1]{\textcolor[rgb]{0.31,0.60,0.02}{#1}}
\newcommand{\VariableTok}[1]{\textcolor[rgb]{0.00,0.00,0.00}{#1}}
\newcommand{\VerbatimStringTok}[1]{\textcolor[rgb]{0.31,0.60,0.02}{#1}}
\newcommand{\WarningTok}[1]{\textcolor[rgb]{0.56,0.35,0.01}{\textbf{\textit{#1}}}}
\usepackage{graphicx}
\makeatletter
\def\maxwidth{\ifdim\Gin@nat@width>\linewidth\linewidth\else\Gin@nat@width\fi}
\def\maxheight{\ifdim\Gin@nat@height>\textheight\textheight\else\Gin@nat@height\fi}
\makeatother
% Scale images if necessary, so that they will not overflow the page
% margins by default, and it is still possible to overwrite the defaults
% using explicit options in \includegraphics[width, height, ...]{}
\setkeys{Gin}{width=\maxwidth,height=\maxheight,keepaspectratio}
% Set default figure placement to htbp
\makeatletter
\def\fps@figure{htbp}
\makeatother
\setlength{\emergencystretch}{3em} % prevent overfull lines
\providecommand{\tightlist}{%
  \setlength{\itemsep}{0pt}\setlength{\parskip}{0pt}}
\setcounter{secnumdepth}{-\maxdimen} % remove section numbering
\ifLuaTeX
  \usepackage{selnolig}  % disable illegal ligatures
\fi
\IfFileExists{bookmark.sty}{\usepackage{bookmark}}{\usepackage{hyperref}}
\IfFileExists{xurl.sty}{\usepackage{xurl}}{} % add URL line breaks if available
\urlstyle{same}
\hypersetup{
  pdftitle={Sintaxe\_TabelaGrafico.R},
  pdfauthor={rafael},
  hidelinks,
  pdfcreator={LaTeX via pandoc}}

\title{Sintaxe\_TabelaGrafico.R}
\author{rafael}
\date{2024-03-27}

\begin{document}
\maketitle

\begin{Shaded}
\begin{Highlighting}[]
\DocumentationTok{\#\#\#\#\#\#\#\#\#\#\#\#\#\#\#\#\#\#\#\#\#\#\#\#\#\#\#\#\#\#\#\#\#\#\#\#\#\#\#\#\#\#\#\#\#\#\#\#\#\#\#\#\#\#\#}
\CommentTok{\# CARREGAR BASE DE DADOS}
\NormalTok{dados }\OtherTok{=} \FunctionTok{read.csv2}\NormalTok{(}\StringTok{"Questionario\_modif.csv"}\NormalTok{, }\AttributeTok{dec=}\StringTok{"."}\NormalTok{)}
\FunctionTok{names}\NormalTok{(dados)}
\end{Highlighting}
\end{Shaded}

\begin{verbatim}
##  [1] "ID"          "Sexo"        "Peso"        "Altura"      "Idade"      
##  [6] "Cursinho"    "Dinheiro"    "Trabalho"    "Cidade"      "Gosta_Est"  
## [11] "Gosta_Curso" "IMC"         "IMC_Cat"
\end{verbatim}

\begin{Shaded}
\begin{Highlighting}[]
\DocumentationTok{\#\#\#\#\#\#\#\#\#\#\#\#\#\#\#\#\#\#\#\#\#\#\#\#\#\#\#\#\#\#\#\#\#\#\#\#\#\#\#\#\#\#\#\#\#\#\#\#\#\#\#\#\#\#\#}
\CommentTok{\# TABELA PARA UMA VARIAVEL QUALITATIVA}
\NormalTok{ftabela }\OtherTok{=} \FunctionTok{table}\NormalTok{(dados}\SpecialCharTok{$}\NormalTok{Gosta\_Est, }\AttributeTok{useNA =} \StringTok{"ifany"}\NormalTok{) }
\NormalTok{ptabela }\OtherTok{=} \FunctionTok{round}\NormalTok{(}\FunctionTok{prop.table}\NormalTok{(ftabela)}\SpecialCharTok{*}\DecValTok{100}\NormalTok{,}\DecValTok{1}\NormalTok{)}
\NormalTok{tabela1 }\OtherTok{=} \FunctionTok{data.frame}\NormalTok{(ftabela,ptabela)}
\NormalTok{tabela1 }\OtherTok{=}\NormalTok{ tabela1[,}\SpecialCharTok{{-}}\DecValTok{3}\NormalTok{]}
\FunctionTok{colnames}\NormalTok{(tabela1) }\OtherTok{\textless{}{-}} \FunctionTok{c}\NormalTok{(}\StringTok{"Gosta\_Est"}\NormalTok{,}\StringTok{"Frequencia"}\NormalTok{,}\StringTok{"Porcentagem"}\NormalTok{)}
\NormalTok{tabela1}
\end{Highlighting}
\end{Shaded}

\begin{verbatim}
##   Gosta_Est Frequencia Porcentagem
## 1   Não sei         14        33.3
## 2       Sim         28        66.7
\end{verbatim}

\begin{Shaded}
\begin{Highlighting}[]
\FunctionTok{write.table}\NormalTok{(tabela1,}\StringTok{"Tabela.csv"}\NormalTok{, }\AttributeTok{sep=}\StringTok{";"}\NormalTok{, }\AttributeTok{dec=}\StringTok{","}\NormalTok{, }\AttributeTok{row.names=}\ConstantTok{TRUE}\NormalTok{)}


\DocumentationTok{\#\#\#\#\#\#\#\#\#\#\#\#\#\#\#\#\#\#\#\#\#\#\#\#\#\#\#\#\#\#\#\#\#\#\#\#\#\#\#\#\#\#\#\#\#\#\#\#\#\#\#\#\#\#\#}
\CommentTok{\# GRAFICO PARA UMA VARIAVEL QUALITATIVA}
\FunctionTok{library}\NormalTok{(ggplot2)}
\FunctionTok{ggplot}\NormalTok{(dados, }\FunctionTok{aes}\NormalTok{(}\AttributeTok{x=}\NormalTok{Cidade )) }\SpecialCharTok{+}
  \FunctionTok{geom\_bar}\NormalTok{(}\AttributeTok{color=}\StringTok{"blue"}\NormalTok{, }\AttributeTok{fill=}\FunctionTok{rgb}\NormalTok{(}\FloatTok{0.1}\NormalTok{,}\FloatTok{0.6}\NormalTok{,}\FloatTok{0.7}\NormalTok{,}\FloatTok{0.7}\NormalTok{) ) }\SpecialCharTok{+}
  \FunctionTok{labs}\NormalTok{(}\AttributeTok{title=}\StringTok{"Gráfico para Cidade"}\NormalTok{,}
       \AttributeTok{x =}\StringTok{"Cidade"}\NormalTok{, }\AttributeTok{y =} \StringTok{"Frequência"}\NormalTok{)}
\end{Highlighting}
\end{Shaded}

\includegraphics{Sintaxe_TabelaGrafico_files/figure-latex/unnamed-chunk-1-1.pdf}

\begin{Shaded}
\begin{Highlighting}[]
\DocumentationTok{\#\#\#\#\#\#\#\#\#\#\#\#\#\#\#\#\#\#\#\#\#\#\#\#\#\#\#\#\#\#\#\#\#\#\#\#\#\#\#\#\#\#\#\#\#\#\#\#\#\#\#\#\#\#\#}
\CommentTok{\# TABELA PARA DUAS VARIAVEIS QUALITATIVAS}
\NormalTok{ftabela2 }\OtherTok{=} \FunctionTok{table}\NormalTok{(dados}\SpecialCharTok{$}\NormalTok{Sexo, dados}\SpecialCharTok{$}\NormalTok{Gosta\_Est, }\AttributeTok{useNA =} \StringTok{"ifany"}\NormalTok{) }
\NormalTok{ptabela2 }\OtherTok{=} \FunctionTok{round}\NormalTok{(}\FunctionTok{prop.table}\NormalTok{(ftabela2,}\DecValTok{1}\NormalTok{)}\SpecialCharTok{*}\DecValTok{100}\NormalTok{,}\DecValTok{1}\NormalTok{)}
\NormalTok{tabela2 }\OtherTok{=} \FunctionTok{data.frame}\NormalTok{(ftabela2,ptabela2)}
\NormalTok{tabela2 }\OtherTok{=}\NormalTok{ tabela2[,}\SpecialCharTok{{-}}\FunctionTok{c}\NormalTok{(}\DecValTok{4}\NormalTok{,}\DecValTok{5}\NormalTok{)]}
\FunctionTok{colnames}\NormalTok{(tabela2) }\OtherTok{\textless{}{-}} \FunctionTok{c}\NormalTok{(}\StringTok{"Sexo"}\NormalTok{,}\StringTok{"Gosta\_Est"}\NormalTok{,}\StringTok{"Frequencia"}\NormalTok{,}\StringTok{"Porcentagem"}\NormalTok{)}
\NormalTok{tabela2}
\end{Highlighting}
\end{Shaded}

\begin{verbatim}
##   Sexo Gosta_Est Frequencia Porcentagem
## 1  Fem   Não sei          5        50.0
## 2 Masc   Não sei          9        28.1
## 3  Fem       Sim          5        50.0
## 4 Masc       Sim         23        71.9
\end{verbatim}

\begin{Shaded}
\begin{Highlighting}[]
\FunctionTok{write.table}\NormalTok{(tabela1,}\StringTok{"Tabela2.csv"}\NormalTok{, }\AttributeTok{sep=}\StringTok{";"}\NormalTok{, }\AttributeTok{dec=}\StringTok{","}\NormalTok{, }\AttributeTok{row.names=}\ConstantTok{TRUE}\NormalTok{)}


\DocumentationTok{\#\#\#\#\#\#\#\#\#\#\#\#\#\#\#\#\#\#\#\#\#\#\#\#\#\#\#\#\#\#\#\#\#\#\#\#\#\#\#\#\#\#\#\#\#\#\#\#\#\#\#\#\#\#\#}
\CommentTok{\# GRAFICO PARA DUAS VARIAVEIS QUALITATIVAS}
\FunctionTok{library}\NormalTok{(ggplot2)}
\FunctionTok{ggplot}\NormalTok{(dados, }\FunctionTok{aes}\NormalTok{(}\AttributeTok{x=}\NormalTok{Sexo, }\AttributeTok{fill=}\NormalTok{Gosta\_Est)) }\SpecialCharTok{+} 
  \FunctionTok{geom\_bar}\NormalTok{(}\AttributeTok{position=}\StringTok{"fill"}\NormalTok{) }\SpecialCharTok{+}
  \FunctionTok{ylab}\NormalTok{(}\StringTok{"Porcentagem"}\NormalTok{)}
\end{Highlighting}
\end{Shaded}

\includegraphics{Sintaxe_TabelaGrafico_files/figure-latex/unnamed-chunk-1-2.pdf}

\begin{Shaded}
\begin{Highlighting}[]
\DocumentationTok{\#\#\#\#\#\#\#\#\#\#\#\#\#\#\#\#\#\#\#\#\#\#\#\#\#\#\#\#\#\#\#\#\#\#\#\#\#\#\#\#\#\#\#\#\#\#\#\#\#\#\#\#\#\#\#}
\CommentTok{\# GRAFICO DE DISPERSAO}
\FunctionTok{library}\NormalTok{(ggplot2)}
\FunctionTok{ggplot}\NormalTok{(dados, }\FunctionTok{aes}\NormalTok{(}\AttributeTok{x=}\NormalTok{Altura, }\AttributeTok{y=}\NormalTok{Peso)) }\SpecialCharTok{+}
  \FunctionTok{geom\_point}\NormalTok{() }
\end{Highlighting}
\end{Shaded}

\includegraphics{Sintaxe_TabelaGrafico_files/figure-latex/unnamed-chunk-1-3.pdf}

\begin{Shaded}
\begin{Highlighting}[]
\FunctionTok{ggplot}\NormalTok{(dados, }\FunctionTok{aes}\NormalTok{(}\AttributeTok{x=}\NormalTok{Altura, }\AttributeTok{y=}\NormalTok{Peso, }\AttributeTok{color=}\NormalTok{Sexo)) }\SpecialCharTok{+}
  \FunctionTok{geom\_point}\NormalTok{() }
\end{Highlighting}
\end{Shaded}

\includegraphics{Sintaxe_TabelaGrafico_files/figure-latex/unnamed-chunk-1-4.pdf}

\begin{Shaded}
\begin{Highlighting}[]
\DocumentationTok{\#\#\#\#\#\#\#\#\#\#\#\#\#\#\#\#\#\#\#\#\#\#\#\#\#\#\#\#\#\#\#\#\#\#\#\#\#\#\#\#\#\#\#\#\#\#\#\#\#\#\#\#\#\#\#}
\CommentTok{\# GRAFICO PARA UMA VARIAVEL QUANTITATIVA}
\CommentTok{\# Histograma}
\FunctionTok{library}\NormalTok{(ggplot2)}
\FunctionTok{ggplot}\NormalTok{(dados, }\FunctionTok{aes}\NormalTok{(}\AttributeTok{x=}\NormalTok{IMC)) }\SpecialCharTok{+} 
  \FunctionTok{geom\_histogram}\NormalTok{(}\AttributeTok{bins=}\DecValTok{7}\NormalTok{)}
\end{Highlighting}
\end{Shaded}

\includegraphics{Sintaxe_TabelaGrafico_files/figure-latex/unnamed-chunk-1-5.pdf}

\begin{Shaded}
\begin{Highlighting}[]
\FunctionTok{ggplot}\NormalTok{(dados, }\FunctionTok{aes}\NormalTok{(}\AttributeTok{x=}\NormalTok{IMC)) }\SpecialCharTok{+} 
  \FunctionTok{geom\_histogram}\NormalTok{(}\AttributeTok{bins =} \DecValTok{7}\NormalTok{, }\AttributeTok{fill =} \StringTok{\textquotesingle{}blue\textquotesingle{}}\NormalTok{, }\AttributeTok{color =} \StringTok{\textquotesingle{}white\textquotesingle{}}\NormalTok{)}
\end{Highlighting}
\end{Shaded}

\includegraphics{Sintaxe_TabelaGrafico_files/figure-latex/unnamed-chunk-1-6.pdf}

\begin{Shaded}
\begin{Highlighting}[]
\DocumentationTok{\#\#\#\#\#\#\#\#\#\#\#\#\#\#\#\#\#\#\#\#\#\#\#\#\#\#\#\#\#\#\#\#\#\#\#\#\#\#\#\#\#\#\#\#\#\#\#\#\#\#\#\#\#\#\#}
\CommentTok{\# TABELA PARA UMA VARIAVEL QUALITATIVA E UMA QUANTITATIVA}
\NormalTok{tabela.medias }\OtherTok{\textless{}{-}} \FunctionTok{aggregate}\NormalTok{(dados}\SpecialCharTok{$}\NormalTok{IMC, }\AttributeTok{by=}\FunctionTok{list}\NormalTok{(dados}\SpecialCharTok{$}\NormalTok{Sexo), }\AttributeTok{FUN=}\StringTok{"mean"}\NormalTok{)}
\FunctionTok{colnames}\NormalTok{(tabela.medias) }\OtherTok{\textless{}{-}} \FunctionTok{c}\NormalTok{(}\StringTok{"IMC"}\NormalTok{,}\StringTok{"Media"}\NormalTok{)}
\NormalTok{tabela.medias}
\end{Highlighting}
\end{Shaded}

\begin{verbatim}
##    IMC    Media
## 1  Fem 23.11532
## 2 Masc 23.85776
\end{verbatim}

\begin{Shaded}
\begin{Highlighting}[]
\DocumentationTok{\#\#\#\#\#\#\#\#\#\#\#\#\#\#\#\#\#\#\#\#\#\#\#\#\#\#\#\#\#\#\#\#\#\#\#\#\#\#\#\#\#\#\#\#\#\#\#\#\#\#\#\#\#\#\#}
\CommentTok{\# SALVAR ARQUIVO EM FORMATO DE RELATORIO}
\CommentTok{\# File {-}\textgreater{} Compile Report...}
\end{Highlighting}
\end{Shaded}


\end{document}
